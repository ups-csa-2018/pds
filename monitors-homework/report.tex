\documentclass[a4paper, 12pt]{article}

\usepackage[cm]{fullpage}
\usepackage{listings}
\usepackage{multicol}
\usepackage{hyperref}
\usepackage{xcolor}

\definecolor{green}{rgb}{0,0.6,0}
\definecolor{gray}{rgb}{0.5,0.5,0.5}
\lstset{
    language=C,
    columns=flexible,
    frame=single,
    xleftmargin=\parindent,
    showstringspaces=false,
    columns=flexible,
    basicstyle={\footnotesize\ttfamily},
    numbers=left,
    numberstyle=\tiny\color{gray},
    keywordstyle=\color{blue},
    commentstyle=\color{green},
    breaklines=true,
    breakatwhitespace=true,
}

\title{Exercise 1 on Hoare Monitors
    \\ \large Parallel and Distributed Systems - Master CSA}
\author{Jules Lamur}

\begin{document}

\maketitle

\begin{abstract}
    This is the first exercise of the exercise sheet that was given as a
    homework:
    \url{https://moodle.univ-tlse3.fr/pluginfile.php/227123/mod_resource/content/3/Monitors.pdf}
\end{abstract}

\section{Introduction}

For the sake of clarity, we consider that the object type \textbf{ring\_buffer}
exists and offers the following methods on its instances:

\begin{itemize}
    \item \textbf{void push(message\_t message)}: enqueues a \textbf{message} in
        the ring buffer ;
    \item \textbf{message\_t pop()}: dequeues a message and returns it ;
    \item \textbf{message\_t front()}: returns the first message to be dequeued
        but do not dequeues it ;
    \item \textbf{bool is\_full()}: true if the ring buffer is full, false
        otherwise ;
    \item \textbf{bool is\_empty()}: true if the ring buffer is empty, false
        otherwise.
    \item \textbf{unsigned int count()}: returns the number of elements
        currently stored inside the buffer.
\end{itemize}

These methods are, of course, \textbf{not thread safe}.
An object of type \textbf{ring\_buffer} is initialized with one parameter, its
size, which is an unsigned integer.

The \textbf{message\_t} type is defined later in every exercise and we suppose
that its copy constructor is sufficient to copy any instance of it.

\pagebreak

\section{Exercise}
\subsection{Version 1}

\subsubsection{Blocking / unblocking conditions}
\begin{itemize}
    \item \textbf{a producer is blocked} when the buffer is full ;
    \item \textbf{a producer is unblocked} when a consumer extracts a message from the
        buffer (the buffer is no longer full) ;
    \item \textbf{a consumer is blocked} when the buffer is empty ;
    \item \textbf{a consumer is unblocked} when a producer inserts a message in the
        buffer (the buffer is no longer empty).
\end{itemize}

\subsubsection{State variables}
\begin{itemize}
    \item \textbf{ring\_buffer buffer}: buffer ring containing messages waiting to be
        consumed.
\end{itemize}

\subsubsection{Condition variables}
\begin{itemize}
    \item \textbf{message\_available}: signaled when the buffer contains at
        least one message ;
    \item \textbf{slot\_available}: signaled when the buffer is not full.
\end{itemize}

\subsubsection{Full code}
\begin{lstlisting}
typedef ... message_t;
unsigned int buf_size = ...;

monitor producer_consumer {
    ring_buffer buffer(buf_size);
    condition_var message_available;
    condition_var slot_available;

    void insert(message_t message) {
        if (buffer.is_full()) {
            slot_available.wait();
        }

        buffer.push(message);
        message_available.signal();
    }

    message_t extract() {
        if (buffer.is_empty()) {
            message_available.wait();
        }

        message_t message = buffer.pop();
        slot_available.signal();

        return message;
    }
}
\end{lstlisting}

\pagebreak

\subsection{Version 2}

\subsubsection{Blocking / unblocking conditions}
\begin{itemize}
    \item \textbf{a producer is blocked} when the buffer is full or when the
        type of the previously inserted message is the same as its own.
    \item \textbf{a producer is unblocked}:
        \begin{itemize}
            \item when a consumer extracts a message from the buffer (the
                buffer is no longer full) and the previously inserted message
                is of a different type ;
            \item or when a producer of a different type inserts a message and
                the buffer is not full ;
        \end{itemize}
    \item \textbf{a consumer is blocked} when the buffer is empty ;
    \item \textbf{a consumer is unblocked} when a producer inserts a message in the
        buffer (the buffer is no longer empty).
\end{itemize}

\subsubsection{State variables}
\begin{itemize}
    \item \textbf{ring\_buffer buffer}: buffer ring containing messages waiting to be
        consumed ;
    \item \textbf{bool last\_inserted\_type}: the type of last inserted
        message.
\end{itemize}

\subsubsection{Condition variables}
\begin{itemize}
    \item \textbf{message\_available}: signaled when the buffer contains at
        least one message ;
    \item \textbf{type\_one\_producer\_turn}: signaled when the producer of type
        one can insert a message ;
    \item \textbf{type\_two\_producer\_turn}: signaled when the producer of type
        two can insert a message.
\end{itemize}

\pagebreak

\subsubsection{Full code}
\begin{lstlisting}
typedef struct {
    ...
    bool type;
} message_t;
unsigned int buf_size = ...;

monitor producer_consumer {
    ring_buffer buffer(buf_size);
    bool last_inserted_type = true;
    condition_var message_available;
    condition_var type_one_producer_turn;
    condition_var type_two_producer_turn;

    // this method is not a protected one (it can be called within
    // another monitor's method without creating a deadlock).
    void signal_producer(bool type) {
        if (type == true) {
            type_two_producer_turn.signal();
        } else {
            type_one_producer_turn.signal();
        }
    }

    void insert(message_t message) {
        if (buffer.is_full() || last_inserted_type == message.type) {
            // we consider that type == true is the producer type one.
            if (message.type == true) {
                type_one_producer_turn.wait();
            } else {
                type_two_producer_turn.wait();
            }
        }

        buffer.push(message);
        last_inserted_type = message.type;

        if (!buffer.is_full()) {
            signal_producer(message_type);
        }

        message_available.signal();
    }

    message_t extract() {
        if (buffer.is_empty()) {
            message_available.wait();
        }

        message_t message = buffer.pop();
        signal_producer(last_inserted_type);

        return message;
    }
}
\end{lstlisting}

\pagebreak

\subsection{Version 3}

\subsubsection{Blocking / unblocking conditions}
\begin{itemize}
    \item \textbf{a producer is blocked} when the buffer is full ;
    \item \textbf{a producer is unblocked} when a consumer extracts a message
        from the buffer (the buffer is no longer full) ;
    \item \textbf{a consumer is blocked} when the next message has a different
        type than its own or the buffer is empty ;
    \item \textbf{a consumer is unblocked}:
        \begin{itemize}
            \item when a consumer extracts a message and the next message has
                the same type as this consumer ;
            \item or when a producer inserts a message of the type of this
                consumer and the buffer was previously empty.
        \end{itemize}
\end{itemize}

\subsubsection{State variables}
\begin{itemize}
    \item \textbf{ring\_buffer buffer}: buffer ring containing messages waiting
        to be consumed.
\end{itemize}

\subsubsection{Condition variables}
\begin{itemize}
    \item \textbf{message\_available[nb\_types]}: signaled when the buffer
        contains at least one message for the given type ;
    \item \textbf{slot\_available}: signaled when the buffer is not full.
\end{itemize}

\pagebreak

\subsubsection{Full code}
\begin{lstlisting}
typedef unsigned int type_t;
typedef struct {
    ...
    type_t type;
} message_t;
unsigned int buf_size = ...;
unsigned int nb_types = ...;

monitor producer_consumer {
    ring_buffer buffer(buf_size);
    condition_var message_available[nb_types];
    condition_var slot_available;

    void insert(message_t message) {
        if (buffer.is_full()) {
            slot_available.wait();
        }

        buffer.push(message);

        if (buffer.size() == 1) {
            message_available[message.type].signal();
        }
    }

    message_t extract(type_t type) {
        if (buffer.is_empty() || buffer.front().type != type) {
            message_available[type].wait();
        }

        message_t message = buffer.pop();
        slot_available.signal();

        if (!buffer.is_empty()) {
            message_available[buffer.front().type].signal();
        }

        return message;
    }
}
\end{lstlisting}

\end{document}
